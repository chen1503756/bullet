
%Use the \path{Demos/MyDemo} instead of \texttt{Demos/MyDemo} otherwise text doesn't wrap
%and runs into the right margin

\chapter{Introduction to Bullet}
%\section{blablaIntroduction}
Bullet Physics is a professional open source collision detection, rigid body and soft body dynamics library. The library is free for commercial use under the \index{zlib license} \href{http://opensource.org/licenses/zlib-license.php}{zlib license}.

\section{Main Features}

\begin{itemize}
	\item Open source C++ code under zlib license and free for any commercial use on all platforms including PLAYSTATION 3, XBox 360, Wii, PC, Linux, Mac OSX, Android and iPhone
	\item Discrete and continuous collision detection including ray and convex sweep test. Collision shapes include concave and convex meshes and all basic primitives
	\item Fast and stable rigid body dynamics constraint solver, vehicle dynamics, character controller and slider, hinge, generic 6DOF and cone twist constraint for ragdolls
	\item Soft Body dynamics for cloth, rope and deformable volumes with two-way interaction with rigid bodies, including constraint support
	\item Maya Dynamica plugin, Blender integration, COLLADA physics import/export support
\end{itemize}

\section{Contact and Support}
\begin{itemize}
	\item Public forum for support and feedback is available at \url{http://bulletphysics.org}
	\item PLAYSTATION 3 licensed developers can download an optimized version for Cell SPU through Sony \href{https://ps3.scedev.net/projects/spubullet}{PS3 Devnet}.
\end{itemize}

\section{What's new}
\subsection{New in Bullet 2.81}
\begin{itemize}
	\item SIMD and Neon optimizations for iOS and Mac OSX, thanks to a contribution from Apple
	\item Rolling Friction using a constraint, thanks to Erin Catto for the idea. See \path{Demos/RollingFrictionDemo/RollingFrictionDemo.cpp}
	\item XML serialization. See \path{Bullet/Demos/BulletXmlImportDemo} and \path{Bullet/Demos/SerializeDemo}
	\item Gear constraint. See \path{Bullet/Demos/ConstraintDemo}.
	\item Improved continuous collision response, feeding speculative contacts to the constraint solver. See \path{Bullet/Demos/CcdPhysicsDemo}
	\item Improved premake4 build system including support for Mac OSX, Linux and iOS
	\item Refactoring of collision detection pipeline using stack allocation instead of modifying the collision object. This will allow better future multithreading optimizations.
\end{itemize}

\section{Building the Bullet SDK and demos}
Windows developers can download the zipped sources of Bullet from \url{http://bullet.googlecode.com}. Mac OS X, Linux and other developers should download the gzipped tar archive.
\subsection{Using premake with Visual Studio}
After unzipping the source code, you can open the \path{Bullet/build} directory and double click on \path{vs2010.bat} to generate Visual Studio 2010 project files and solution. Just open \path{Bullet/build/vs2010/0BulletSolution.sln}
\subsection{Using premake with Xcode for Mac OSX or iOS}

\subsection{Using cmake}

\subsection{Using autotools}
